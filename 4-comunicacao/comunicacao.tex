\documentclass[t]{beamer}
\usepackage[portuguese]{babel}
\usepackage[utf8]{inputenc}
\usetheme{Berkeley}
\usecolortheme{seahorse}

\addto\captionsportuguese{
	\renewcommand{\figurename}{Fig.}
	\renewcommand{\tablename}{Tab.}
}

\title{Meios de Comunicação}
\subtitle{Como podemos trocar dados entre dispositivos.}

\AtBeginSection[]
{
	\begin{frame}
	\frametitle{Sumário}
	\tableofcontents[currentsection]
\end{frame}
}

\begin{document}

\frame{\titlepage}

\begin{frame}
\frametitle{Sumário}
\tableofcontents
\end{frame}

\section{Comunicação}

\begin{frame}{Como nos comunicamos}
\begin{itemize}
	\item Remetente
	\item Mensagem
	\item Emissor
	\item Código
	\item Meio
	\item Ruído
	\item Receptor
	\item Destinatário
	\item ..
\end{itemize}
\end{frame}


\begin{frame}{Como nos comunicamos}
Modelo de comunicação de Shannon e Weaver
\begin{figure}
	\includegraphics[width=\linewidth]{Shannon-Weaver_model}
	
	\scriptsize Fonte: Claude Shannon e Warren Weaver, "The Mathematical Theory of Communication", University of Illinois Press, Urbana, IL, 1949
\end{figure}
\end{frame}

\section{Tecnologias}

\begin{frame}{Tecnologias para comunicação}
Modo de comunicação
\begin{itemize}
	\item Curta distância
	\item Longa distância
	\item Híbridas?!
\end{itemize}
Tipo de comunicação
\begin{itemize}
	\item Com fio
	\item Sem fio
\end{itemize}
Tipo de rede
\begin{itemize}
	\item LAN
	\item WAN
\end{itemize}
\end{frame}


\begin{frame}{Opções de comunicação}
Algumas opções disponíveis
\begin{itemize}
	\item Cabos: Jumpers, Ethernet e outros.
	\item Infra-vermelho
	\item NFC (Near Field Communication)
	\item Bluetooth v2, v3, v4, v5, A2DP e BLE
	\item Rádio-Frequência
	\item ZigBee
	\item WiFi a, b, g, n, ac e WiFi Direct
	\item 2G, 3G, 4G, 5G
	\item LoRa, Sigfox, NB-IoT, EC-GSM-IoT, LTE Cat-M1, RPMA e vários outros
\end{itemize}

\end{frame}

\begin{frame}{Opções de comunicação}
Infra-vermelho
\begin{itemize}
	\item 1m ou mais
	\item 15 a 30 graus para transmissão
	\item 15 graus para recepção
	\item 2.400 a 115.200 bps no início
	\item Atualmente suporta até 10 Gbps para serviços de broadcasting
\end{itemize}

\end{frame}

\begin{frame}{Opções de comunicação}
NFC
\begin{itemize}
	\item 4 a 10~cm
	\item 0.4Mbps
\end{itemize}
\end{frame}

\begin{frame}{Opções de comunicação}
Bluetooth
\begin{itemize}
	\item 2.4 GHz ou 5GHz
	\item Saída 36.3 a 585.6 kbps com link de 1Mbps no padrão inicial
	\item Recomendação de 10~m, com máximo de 100~m e distância para comunicação.
	\item Hoje suporta 2Mbps ou mais caso use 5GHz
	\item Também permite alcance e mais de 400~m em alguns casos com alto consumo de energia.
	\item Bluetooth Low Energy~(BLE) é otimizado para IoT
\end{itemize}
\end{frame}


\begin{frame}{Opções de comunicação}
ZigBee
\begin{itemize}
	\item 20 a 250~kbps
	\item 1 a 100~m
	\item Arquitetura de rede tipo Mesh
\end{itemize}
\end{frame}

\begin{frame}{Opções de comunicação}
WiFi
\begin{figure}
	\includegraphics[width=\linewidth]{wirelessstandards}
\end{figure}
\end{frame}

\begin{frame}{Opções de comunicação}
Comparativo
\begin{figure}
	\includegraphics[width=\linewidth]{chart-range-vs-data}
\end{figure}
\end{frame}

\section{Roteamentos}

\begin{frame}{Roteamentos}
Tipos de roteamentos
\begin{itemize}
	\item Unicast
	\item Multicast
	\item Broadcast
	\item Anycast
	\item Geocast
\end{itemize}
\end{frame}

\begin{frame}{Roteamentos}
Unicast
\begin{figure}
	\includegraphics[width=\linewidth]{Unicast}
\end{figure}
\end{frame}

\begin{frame}{Roteamentos}
Multicast
\begin{figure}
	\includegraphics[width=\linewidth]{Multicast}
\end{figure}
\end{frame}

\begin{frame}{Roteamentos}
Broadcast
\begin{figure}
	\includegraphics[width=\linewidth]{Broadcast}
\end{figure}
\end{frame}

\begin{frame}{Roteamentos}
Anycast
\begin{figure}
	\includegraphics[width=\linewidth]{Anycast-BM}
\end{figure}
\end{frame}


\begin{frame}{Roteamentos}
Geocast
\begin{figure}
	\includegraphics[width=\linewidth]{Geocast}
\end{figure}
\end{frame}

\section{Protocolos}

\begin{frame}{Protocolos}
O que são protocolos?!
\begin{itemize}
	\item Conjunto de regras
	\begin{itemize}
		\item Sintaxe
		\item Semântica
		\item Sincronização
		\item Correção de erros
	\end{itemize}
\end{itemize}
\bigskip
\begin{quotation}
	``Protocolos são para a comunicação o que as linguagens de programação são para a computação.'' (Comer, 2000)
\end{quotation}
\end{frame}

\begin{frame}{Protocolos}
Exemplos
\begin{itemize}
	\item MAC (Medium Access Control)
	\item IP (Internet Protocol)
	\item TCP (Transmission Control Protocol)
	\item UDP (User Datagram Protocol)
	\item HTTP (Hypertext Transfer Protocol)
	\item MQTT (Message Queuing Telemetry Transport)
	\item CoAP (
	Constrained Application Protocol)
\end{itemize}
\end{frame}

\begin{frame}{Protocolos}
MQTT
\begin{figure}
	\includegraphics[width=0.8\linewidth]{mqtt_publisher_subscriber}
\end{figure}
\end{frame}

\begin{frame}{Protocolos}
CoAP
\begin{figure}
	\includegraphics[width=\linewidth]{CoAP-Architecture}
\end{figure}
\end{frame}

\section{Serviços}

\begin{frame}{Serviços para troca de informações}
Tipos de serviços para troca de informações:
\begin{itemize}
	\item Serviços locais
	\item Web services
	\item Cloud services
\end{itemize}
\end{frame}

\begin{frame}{Serviços para troca de informações}
Serviços locais
\begin{itemize}
	\item Unicast, Broadcast, Multicast
	\item mDNS (Bonjour, zeroconf)
	\item UPnP (Universal Plug and Play) 
	\item DLNA (Digital Living Network Alliance)
\end{itemize}
\end{frame}

\begin{frame}{Serviços para troca de informações}
Web Services
\begin{itemize}
	\item Usados na integração com aplicações Web
	\item Tecnologias relacionadas:
	\begin{itemize}
		\item Uso de padrões Json, XML, SOAP, WSDL e UDDI
		\item AJAX
		\item Arquitetura REST (Representational State Transfer)
		\item Framework Swagger	
	\end{itemize}	
\end{itemize}
\end{frame}

\begin{frame}{Serviços para troca de informações}
Cloud Services
\begin{itemize}
	\item Usados na integração com a nuvem
	\item Tecnologias relacionadas:
	\begin{itemize}
		\item SaaS (Software as a Service)
		\item PaaS (Platform as a Service)
		\item IaaS (Infrastructure as a Service)
		\item MaaS (Monitoring as a Service)
		\item CaaS (Communication as a Service)
		\item DaaS (Data as a Service)
		\item XaaS (Anything as a Service)
	\end{itemize}	
\end{itemize}
\end{frame}

\frame{\titlepage}

\end{document}
